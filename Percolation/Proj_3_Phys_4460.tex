\documentclass[11pt]{article}
\usepackage{verbatim}
\usepackage{listings}
\usepackage{graphicx}
\usepackage{a4wide}
\usepackage{color}
\usepackage{amsmath}
\usepackage{amssymb}
\usepackage[dvips]{epsfig}
\usepackage[T1]{fontenc}
\usepackage{cite} % [2,3,4] --> [2--4]
\usepackage{shadow}
\usepackage{hyperref}
\usepackage{physics}
\usepackage{url}
\usepackage{tikz}
\usepackage{subcaption}
\usepackage[utf8]{inputenc}
\usepackage{booktabs} % Allows the use of \toprule, \midrule and \bottomrule in tables
\usepackage{listings}  
\usepackage{minted}

\renewcommand\thesection{Q (\alph{section})}


\usetikzlibrary{arrows, shapes}

\setcounter{tocdepth}{2}




\title{ FYS-4460 \\ Project 3 }
\author{Gullik Vetvik Killie}
		

\begin{document}
\lstset{language=Matlab}

\maketitle

\tableofcontents

\section{}
	Using these features, you should make a program to calculate P(p, L) for various
	p. Hint: you can use either BoundingBox or intersect and union to find
	the spanning cluster. How robust is your algorithm to changes in boundary
	conditions? Could you do a rectangular grid where Lx \( >> \) Ly? Could you do
	a more complicated set of boundaries? Can you think of a simple method to
	ensure that you can calculate P for any boundary geometry?
	\\
	\textbf{Answer}
	
	\inputminted{Matlab}{ percolation.m }

\section{}
	We know that when \(p > p_c\), the probability \(P(p, L)\) for a given site to belong to
	the percolation cluster, has the form \(P(p,L) \propto |p-p_c|^\beta\)

	Use your program to find an expression for \(\beta \). For this you may need that
	pc = 0.59275.
	\\
	\textbf{Answer}
	Since \(P\) is proportional to \( |p-p_c|^\beta \) a logarithmic regression can be used.

	\begin{align}
		P(p,L) &\propto |p-p_c|^\beta \\
		\log{(P(p,L))} &\propto \beta \log{(p-p_c)}
	\end{align}


\end{document}